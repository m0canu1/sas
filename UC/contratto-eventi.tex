\documentclass[12pt]{extarticle}


\begin{document}

\title{Contratti}
\date{}
\maketitle

\subsection*{Pre-condizione generale}

\begin{itemize}
  \item L'attore dev'essere identificato come un'istanza \textbf{org} di Organizzatore
\end{itemize}

\section*{1. creaEvento(\underline{titolo}? : testo)}

\textbf{Pre-condizione}
\begin{itemize}
  \item Pre-condizione generale.
\end{itemize}
\textbf{Post-condizione}
\begin{itemize}
  \item è stata creata un'istanza \textbf{e} di $Evento$
  \item (se è stato specificato un \textbf{\underline{titolo}}) \textit{e.titolo} = \textbf{\underline{titolo}}
  \item \textit{org} \textbf{è proprietario} di \textbf{e}
  \item \textit{e.pubblicato} = no
\end{itemize}

\subsection*{1a.1 scegliEvento(\underline{evento}: Evento)}
\textbf{Pre-condizione}
\begin{itemize}
  % \item Pre-condizione generale.
  \item ---
  % \item \textit{org} \textbf{è proprietario} di \textbf{e}
\end{itemize}
\textbf{Post-condizione}
\begin{itemize}
  % \item viene restituita l'istanza \textbf{e} dell'\textit{Evento}
  \item ---
\end{itemize}

NdR: Attenzione: gli errori che può dare un’operazione non sono pre-condizioni. Se una pre-condizione è falsa l’operazione non può avvenire. Nel nostro caso l’operazione avviene ma può dare errore. Se avessimo delle post-condizioni potremmo dire che in alcuni casi quelle post-condizioni non si verificano, ma in questo caso non ci sono nemmeno post-condizioni.

% \subsection*{1b. eliminaEvento(\underline{evento}: Evento)}
\subsection*{1b.1 eliminaEvento(\underline{evento}: Evento)}

\textbf{Pre-condizione}
\begin{itemize}
  % \item Pre-condizione generale
  \item ---
  % \item \textit{org} è proprietario di \textbf{e}
\end{itemize}
\textbf{Post-condizione} Se \textit{org} è \textbf{proprietario} di \underline{evento}
\begin{itemize}
  % \item \textit{e.annullato} = sì (non viene eliminato completamente perché si vogliono tenere i dati??)
  \item il Foglio Riepilogativo \textit{foglio\_riepilogativo} che si riferisce a \underline{evento} è eliminato
  \item la Scheda \textit{scheda} associata a \underline{evento} è eliminata
  \item lo Chef \textit{chef} incaricato dell'\underline{evento} è reso (disponibile???????)
  \item Ogni istanza di Membro\_del\_Personale \textit{personale} previsto per \underline{evento} è reso (disponibile???????)
\end{itemize}


\section*{2. compilaScheda(\underline{evento}: Evento, \underline{data}: data, \underline{luogo}:testo, \underline{numero\_partecipanti?}: numero, \underline{chef}: Chef, \underline{personale} Membro\_del\_Personale)}

\textbf{Pre-condizione}
\begin{itemize}
  \item è in corso la creazione di un $Evento$ \textbf{e}
\end{itemize} 
\textbf{Post-condizione} 
\begin{itemize}
  \item è stata creata un'istanza \textbf{s} di $Scheda$
  \item \textbf{s}.data = data
  \item \textbf{s}.luogo = luogo
  \item \textbf{s}.numero\_partecipanti = numero\_partecipanti
  \item \textbf{e} contiene \textbf{s}
\end{itemize}

\subsection*{2a. modificaScheda(scheda: Scheda, data?: data, luogo: testo, numero\_partecipanti?: numero)}

\textbf{Pre-condizione}
\begin{itemize}
  \item è in corso la creazione di un $Evento$ \textbf{e}
\end{itemize} 
\textbf{Post-condizione}
\begin{itemize}
  \item una istanza \textbf{s} di $Scheda$ è modificata
  \item (se presente) \textbf{s}.data = data
  \item (se presente) \textbf{s}.luogo = luogo
  \item (se presente) \textbf{s}.numero\_partecipanti = numero\_partecipanti
\end{itemize} 


\section*{3. assegnaChef(evento: Evento, chef: Chef)}

\textbf{Pre-condizione}
\begin{itemize}
  \item è in corso la creazione di un $Evento$ \textbf{e}
\end{itemize}
\textbf{Post-condizione}
\begin{itemize}
  \item è stata creata un'istanza \textbf{c} di  $Chef$
  \item \textbf{c} è associato all'evento \textbf{e}
\end{itemize}


\subsection*{3a. modificaChef(c: Chef, \textit{chef\_new}: Chef)}

\textbf{Pre-condizione}
\begin{itemize}
  \item lo chef \textbf{c} esiste
  \item è in corso la creazione di un $Evento$ \textbf{e}
\end{itemize}
\textbf{Post-condizione}
\begin{itemize}
  \item lo chef \textbf{chef\_new} è associato all'evento
\end{itemize}


\section*{4. assegnaPersonale()}

\textbf{Pre-condizione}
\begin{itemize}
  \item è in corso la creazione di un $Evento$ \textbf{e}
\end{itemize}
\textbf{Post-condizione}
\begin{itemize}
  \item è stata creata una istanza \textbf{m} di \textit{Membro\_del\_Personale}
  \item \textbf{m} è associato all'evento
\end{itemize}


\subsection*{4a. aggiungiPersonale(membro: \textit{Membro\_del\_Personale})}

\textbf{Pre-condizione}
\begin{itemize}
  \item è in corso la creazione di un $Evento$ \textbf{e}
  \item esiste una istanza \textbf{m} di \textit{Membro\_del\_Personale}
\end{itemize}
\textbf{Post-condizione}
\begin{itemize}
  \item il membro del personale è aggiunto all'evento
\end{itemize}


\subsection*{4b. eliminaPersonale(membro: \textit{Membro\_del\_Personale})}

\textbf{Pre-condizione}
\begin{itemize}
  \item è in corso la creazione di un $Evento$ \textbf{e}
  \item membro è parte del personale dell'evento
\end{itemize}
\textbf{Post-condizione}
\begin{itemize}
  \item membro è eliminato dal personale dell'evento
\end{itemize}


\section*{5. scriviNote(evento: Evento, nota: testo)}

\textbf{Pre-condizione}
\begin{itemize}
  \item l'\textbf{evento} è terminato
\end{itemize}
\textbf{Post-condizione}
\begin{itemize}
  \item \textbf{evento}.note = nota
\end{itemize}


\subsection*{(2-5)a. eliminaEvento(evento: Evento, penale: si/no)}

\textbf{Pre-condizione}
\begin{itemize}
  \item è in corso la creazione di un $Evento$ \textbf{e}
\end{itemize}
\textbf{Post-condizione}
\begin{itemize}
  \item \textbf{e}.annullato = sì
  \item \textbf{e}.penale = penale
  \item la compilazione è interrotta
\end{itemize}  

\end{document}


