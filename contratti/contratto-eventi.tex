\documentclass[12pt]{extarticle}


\begin{document}

\title{Contratti}
\date{}
\maketitle

\subsection*{Pre-condizione generale}

\begin{itemize}
  \item L'attore dev'essere identificato come un'istanza \textbf{org} di Organizzatore
\end{itemize}

\section*{1. creaEvento()}

\textbf{Pre-condizione}
\begin{itemize}
  \item Pre-condizione generale.
\end{itemize}
\textbf{Post-condizione}
\begin{itemize}
  \item è stata creata un'istanza \textbf{e} di $Evento$
  \item \textit{org} \textbf{è proprietario} di \textbf{e}
  \item \textit{e.pubblicato} = no
\end{itemize}

\subsection*{1a.1 scegliEvento(evento: Evento)}
\textbf{Pre-condizione}
\begin{itemize}
  % \item Pre-condizione generale.
  \item ---
  % \item \textit{org} \textbf{è proprietario} di \textbf{e}
\end{itemize}
\textbf{Post-condizione}
\begin{itemize}
  % \item viene restituita l'istanza \textbf{e} dell'\textit{Evento}
  \item ---
\end{itemize}

%%NdR: Attenzione: gli errori che può dare un’operazione non sono pre-condizioni. Se una pre-condizione è falsa l’operazione non può avvenire. Nel nostro caso l’operazione avviene ma può dare errore. Se avessimo delle post-condizioni potremmo dire che in alcuni casi quelle post-condizioni non si verificano, ma in questo caso non ci sono nemmeno post-condizioni.

\subsection*{1b.1 eliminaEvento(evento: Evento)}

\textbf{Pre-condizione}
\begin{itemize}
  % \item Pre-condizione generale
  \item ---
  % \item \textit{org} è proprietario di \textbf{e}
\end{itemize}
\textbf{Post-condizione} Se \textit{org} è \textbf{proprietario} di evento
\begin{itemize}
  % \item \textit{e.annullato} = sì (non viene eliminato completamente perché si vogliono tenere i dati??)
  \item il Foglio Riepilogativo \textit{foglio\_riepilogativo} che si riferisce a evento è eliminato
  \item la Scheda \textit{scheda} associata a evento è eliminata
  \item lo Chef \textit{chef} incaricato dell'evento non è più assegnato all'evento
  \item Ogni istanza di Membro\_del\_Personale \textit{personale} previsto per evento non è più associata all'evento
\end{itemize}

\section*{2. compilaScheda(e: Evento, data: Data, luogo:testo, numero\_partecipanti: numero)}

\textbf{Pre-condizione}
\begin{itemize}
  \item è in corso la creazione di un $Evento$ \textbf{e}
\end{itemize} 
\textbf{Post-condizione} 
\begin{itemize}
  \item è stata creata un'istanza \textbf{s} di $Scheda$
  \item \textbf{s}.data = data
  \item \textbf{s}.luogo = luogo
  \item \textbf{s}.numero\_partecipanti = numero\_partecipanti
  \item \textbf{e} contiene \textbf{s}
\end{itemize}

 

\section*{3. assegnaChef(evento: Evento, chef: Chef)}

\textbf{Pre-condizione}
\begin{itemize}
  \item è in corso la creazione di un $Evento$ \textbf{e}
\end{itemize}
\textbf{Post-condizione}
\begin{itemize}
  \item è stata creata un'istanza \textbf{c} di  $Chef$
  \item \textbf{c} è associato all'evento \textbf{e}
\end{itemize}

 

\section*{4. assegnaPersonale(s: Scheda, m: Membro del personale)}

\textbf{Pre-condizione}
\begin{itemize}
  \item è in corso la creazione di un $Evento$ \textbf{e}
\end{itemize}
\textbf{Post-condizione}
\begin{itemize}
  \item \textbf{s}, associata all'evento \textbf{e} è modificata
  \item \textbf{s}.numero\_partecipanti = numero\_partecipanti
\end{itemize} 

\subsection*{4a.1 aggiungiRuoloPersonale(m: \textit{Membro del Personale}, ruolo: Ruolo)}
\textbf{Pre-condizione}
\begin{itemize}
  \item è in corso la creazione di un $Evento$ \textbf{e}
  \item m è parte del personale dell'evento
\end{itemize}
\textbf{Post-condizione}
\begin{itemize}
  \item il membro del personale \textbf{m} dell'evento \textbf{e} è associato al ruolo \textbf{r} per l'evento
\end{itemize}


\subsection*{4b.1 rimuoviPersonale(m: \textit{Membro del Personale})}
\textbf{Pre-condizione}
\begin{itemize}
  \item è in corso la creazione di un $Evento$ \textbf{e}
  \item m è parte del personale dell'evento
\end{itemize}
\textbf{Post-condizione}
\begin{itemize}
  \item m è eliminato dal personale dell'evento
\end{itemize}

\subsection*{4c.1 rimuoviRuoloPersonale(m: \textit{Membro del Personale})}
\textbf{Pre-condizione}
\begin{itemize}
  \item è in corso la creazione di un $Evento$ \textbf{e}
  \item m è parte del personale dell'evento
  \item m è associato ad un ruolo \textbf{r} per l'evento
\end{itemize}
\textbf{Post-condizione}
\begin{itemize}
  \item il membro del personale \textbf{m} non è più associato al ruolo \textbf{r} per l'evento \textbf{e} 
\end{itemize}

\subsection*{(2-4)a.1 modificaData(s: Scheda, data: data)}

\textbf{Pre-condizione}
\begin{itemize}
  \item è in corso la creazione di un $Evento$ \textbf{e}
\end{itemize} 
\textbf{Post-condizione}
\begin{itemize}
  \item \textbf{s}, associata all'evento \textbf{e} è modificata
  \item \textbf{s}.data = data
\end{itemize} 

\subsection*{(2-4)b.1 modificaLuogo(s: Scheda, luogo: Testo)}

\textbf{Pre-condizione}
\begin{itemize}
  \item è in corso la creazione di un $Evento$ \textbf{e}
\end{itemize} 
\textbf{Post-condizione}
\begin{itemize}
  \item \textbf{s}, associata all'evento \textbf{e} di $Scheda$ è modificata
  \item \textbf{s}.luogo = luogo
\end{itemize} 


\subsection*{(2-4)c.1 modificaNPartecipanti(s: Scheda, numero\_partecipanti: numero)}
\textbf{Pre-condizione}
\begin{itemize}
  \item è in corso la creazione di un $Evento$ \textbf{e}
\end{itemize} 
\textbf{Post-condizione}
\begin{itemize}
  \item \textbf{s}, associata all'evento \textbf{e} è modificata
  \item \textbf{s}.numero\_partecipanti = numero\_partecipanti
\end{itemize}

\subsection*{(3-4)a.1 modificaChef(s: Scheda, c: Chef)}
\textbf{Pre-condizione}
\begin{itemize}
  \item è in corso la creazione di un $Evento$ \textbf{e}
\end{itemize} 
\textbf{Post-condizione}
\begin{itemize}
  \item \textbf{s}, associata all'evento \textbf{e}, è modificata
  \item Lo chef c è assegnato a \textbf{s}
\end{itemize}

\section*{5. pubblicaEvento(e: Evento)}
\textbf{Pre-condizione}
\begin{itemize}
  \item è in corso la creazione di un $Evento$ \textbf{e}
\end{itemize}
\textbf{Post-condizione}
\begin{itemize}
    \item \textbf{e}.pubblicato = si
\end{itemize}

\subsection*{(2-5)a.1 eliminaEvento(e: Evento)}

\textbf{Pre-condizione}
\begin{itemize}
  \item è in corso la creazione di un $Evento$ \textbf{e}
\end{itemize}
\textbf{Post-condizione}
\begin{itemize}
  \item l'evento \textbf{e} è eliminato
  \item la compilazione è interrotta
\end{itemize}  


\subsection*{(2-5)b.1 annullaEvento(e: Evento)}

\textbf{Pre-condizione}
\begin{itemize}
  \item è in corso la creazione di un $Evento$ \textbf{e}
\end{itemize}
\textbf{Post-condizione}
\begin{itemize}
  \item \textbf{e}.annullato = si;
\end{itemize}  


\subsection*{(2-5)b.2 impostaPenale(e: Evento, p: si/no)}
\textbf{Pre-condizione}
\begin{itemize}
  \item è in corso la creazione di un $Evento$ \textbf{e}
  \item \textbf{e} è stato annullato
\end{itemize}
\textbf{Post-condizione}
\begin{itemize}
  \item \textbf{e}.penale = p;
  \item la compilazione è interrotta
\end{itemize}  

\section*{(2-5)c.1 scriviNota(e: Evento, nota: testo)}
\textbf{Pre-condizione}
\begin{itemize}
  \item L'evento \textbf{e} è terminato
\end{itemize}
\textbf{Post-condizione}
\begin{itemize}
  \item \textbf{e}.note = nota
\end{itemize}


\end{document}
